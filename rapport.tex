\documentclass{article}

\usepackage[utf8]{inputenc}

\title{Travail pratique \#1 - IFT-2245}
\author{Nicolas Hurtubise et Guillaume Riou}

\begin{document}

\maketitle

\newpage

\section{Intro}

Le shell ch constitue un shell de base minimaliste implémentant :

\begin{enumerate}
\item L'appel de programmes exécutables (présents dans le PATH ou via un chemin absolu ou relatif) avec arguments
\item L'expansion de `*` dans les arguments par la liste des fichiers et dossiers présents dans le dossier courrant
\item Le changement de dossier via la commande ``built-in'' cd
\item La redirection de flux d'entrée et de sortie vers d'autres programmes (des ``pipes'') ainsi que vers des fichiers (via \textless et \textgreater)
\end{enumerate}

\section{Stratégie d'implantation}

En début de projet, nous avons considéré apprendre de nouveaux outils d'analyse sémantique et lexicale tels que \texttt{yacc} et \texttt{lex}, suggestion faite par le merveilleux et talentueux démonstrateur, Alexandre St-Louis-Fortier. Étant données les contraintes de temps et le niveau de complexité relativement restreint du projet, nous avons plutôt opté pour des fonctions de la librairie string.h, plus simples d'utilisation pour des néophytes.


Notre stratégie est de séparer les tokens en entré sur les espaces (caractère ASCII 32) et de les accumuler jusqu'à obtenir une redirection de flux ('\textless', '\textgreater' ou '\textbar') ou le caractère de fin de chaîne '\\0'. Le premier token est considéré comme le nom de la commande, les suivants comme les arguments qui lui seront passés.

Au début de l'implantation, nous avons opté pour l'utilisation de la fonction strsep qui est en quelque sorte un remplacement pour strtok. En effet, cette fonction permet de gérer plus facilement les délimiteurs multiples.

Chaque commande entrée génère utilise la sortie de la commande précédente (ou \texttt{stdin} par défaut) en guise d'entrée et envoie sa sortie vers la commande suivante (ou vers \texttt{stdout} par défaut). Il est possible de donner à une commande un fichier pour \texttt{stdin} ou \texttt{stdout}, respectivement grâce aux opérateurs \textless et \textgreater.

\section{Difficulté rencontrées, limitations et perspectives futures}
Étant donné que le programme que nous avons écrit est en C, nous avons rencontrés des difficultés sous la forme de bugs cryptiques et difficiles à tracer. C'est dans ces situations que nous avons appris à apprécier Valgrind qui donne souvent un bien meilleur apperçu de la situation que la gestion d'erreur rudimentaire de C. 

Une des plus grande difficulté que nous avons du gérer durant la réalisation de ce projet est celle de concevoir le flot d'un programme se divisant plusieurs fois. En effet, concevoir un programme qui exécute simultanément plusieurs branches de code est un défi que nous n'avions jamais encore relevés. 

Vu que notre programme a été développé sur un intervalle de temps limité dans le cadre d'un cours, il est certain qu'il possède des limitations qu'un shell mature n'a pas. La méthode d'analyse de l'entrée utilisateur que nous avons utilisés constite une de ces limitations. Le nombre maximum de caractères dans une commande est 128 (une constante dans le code qu'il est possible de changer par une recompilation), après ce nombre, le programme considère le reste comme une autre commande séparée.

Notre opérateur '*' ne peut être utilisé que seul pour faire la liste de tous les fichiers contenus dans le dossier courrant. Notre gestion des délimiteurs multiple pourrait aussi être améliorée : les délimiteurs multiples autour des arguments d'une commande sont supportés mais ceux autours d'un opérateurs de redirection ne le sont pas.

Une autre limitation est l'impossibilité d'échapper des charactères dans un argument. Il est ainsi impossible de faire référence à des dossiers dont le nom comprend des espaces.

Il serait bien entendu intéressant de rendre notre shell plus complet en le modifiant pour ajouter des structures de contrôles comme des conditions, des boucles et des opérations booléennes telles que \&\& ou \textbar\textbar se basant sur les valeurs de retour des processus. Il serait aussi intéressant d'y ajouter certaines fonctionnalités pratique de bash telles que la possibilité de définir des alias ou de pouvoir exécuter un fichier de configuration (.chrc) au démarage du shell.

\newpage

\section{Outils utilisés}

Plusieurs outils nous ont été utiles au cours de la réalisation de ce projet : 

\begin{enumerate}
\item \texttt{make} et \texttt{gcc} pour la compilation du code et son automatisation
\item \texttt{valgrind} pour l'élimination des fuites de mémoire
\item \texttt{git} pour le partage aisé de code source
\item \LaTeX pour le formattage de ce rapport
\item sans oublier l'excellent éditeur de texte \texttt{Emacs}, pour la coloration syntaxique, l'édition de code, l'indentation automatique et le divertissement lorsque le moral était bas\footnote{On peut difficilement s'ennuyer avec zone-mode, life-mode, pong, tetris, ... et doctor-mode a su nous prêter une oreille d'une grande aide dans les moments difficiles.}
\end{enumerate}

\section{Conclusion}

L'implantation de ce shell nous a permi d'en apprendre plus sur les bases de la programmation en C sous Unix, notamment sur l'utilisation des pipes et des redirections de flux d'entrée/sortie.

L'une des plus grandes révélations a certainement été le côté omniprésent des fichiers dans ce genre de systèmes. En effet, en définissant \texttt{stdin} et \texttt{stdout} de la même façon que les fichiers réguliers, des problèmes tels que la redirection de flux depuis/vers un fichier sont grandement réduits.

\end{document}
