\documentclass{article}

\usepackage[utf8]{inputenc}
\usepackage{amsfonts}            %For \leadsto
\usepackage{amsmath}             %For \text
\usepackage{fancybox}            %For \ovalbox

\title{Travail pratique \#1}
\author{IFT-2245}

\begin{document}

\maketitle

{\centering \ovalbox{\large ¡¡ Dû le 28 janvier à 23h59 !!} \\}

\newcommand \mML {\ensuremath\mu\textsl{ML}}
\newcommand \kw [1] {\textsf{#1}}
\newcommand \id [1] {\textsl{#1}}
\newcommand \punc [1] {\kw{`#1'}}
\newcommand \str [1] {\texttt{"#1"}}
\newenvironment{outitemize}{
  \begin{itemize}
  \let \origitem \item \def \item {\origitem[]\hspace{-18pt}}
}{
  \end{itemize}
}
\newcommand \Align [2][t] {
  \begin{array}[#1]{@{}l}
    #2
  \end{array}}

\section{Survol}

Ce TP vise à vous familiariser avec l'environnement de développement d'un
système d'exploitation tel que Linux et les outils typiquement utilisés dans
ce contexte.
Les étapes de ce travail sont les suivantes:
\begin{enumerate}
\item Parfaire sa connaissance de C et Make et POSIX.
\item Lire et comprendre cette donnée.  Cela prendra probablement une partie
  importante du temps total.
\item Lire, trouver, et comprendre les parties importantes du code fourni.
\item Compléter le code fourni.
\item Écrire un rapport.  Il doit décrire \textbf{votre} expérience pendant
  les points précédents: problèmes rencontrés, surprises, choix que vous
  avez dû faire, options que vous avez sciemment rejetées, etc...  Le
  rapport ne doit pas excéder 5 pages.
\end{enumerate}

Ce travail est à faire en groupes de 2 étudiants.  Le rapport, au format
\LaTeX\ exclusivement (compilable sur \texttt{frontal.iro}) et le code sont
à remettre par remise électronique avant la date indiquée.  Aucun retard ne
sera accepté.  Indiquez clairement votre nom au début de chaque fichier.

Si un étudiant préfère travailler seul, libre à lui, mais l'évaluation de
son travail n'en tiendra pas compte.  Si un étudiant ne trouve pas de
partenaire, il doit me contacter au plus vite.  Des groupes de 3 ou plus
sont \textbf{exclus}.

\newpage
\section{CH: un shell pour les hélvètes}

Vous allez devoir implanter une ligne de commande, similaire
à \texttt{/bin/sh}, qui sait:
\begin{enumerate}
\item Démarrer des processus externes: être capable d'exécuter des commandes
  comme ``\texttt{cat Makefile}''.
\item Expansion d'arguments: être capable d'exécuter des commandes comme
  ``\texttt{echo *}''.
\item Rediriger les entrées et sorties de ces processus: être capable
  d'exécuter des commandes comme ``\texttt{cat <Makefile >foo}''.
\item Connecter ces processus via des \emph{pipes} pour faire des
  \emph{pipelines}: être capable d'exécuter des commandes comme
  ``\texttt{find -name Makefile | xargs grep ch}''.
\end{enumerate}

Tout cela bien sûr sans réutiliser un autre shell mais en utilisant les
primitives du système d'exploitation, telles que \texttt{exec},
\texttt{fork}, \texttt{dup2}, \texttt{pipe}, \texttt{readdir}, ...

Le programme doit être compilable et exécutable sur \texttt{frontal.iro}.
Cela ne vous empêche pas bien sûr de le développer sur un système différent,
e.g. sous Windows avec Cygwin, mais assurez-vous que le résultat fonctionne
\emph{aussi} sur \texttt{frontal.iro}.

\section{Cadeaux}

Vous recevez en cadeau de bienvenue les fichiers \texttt{Makefile},
\texttt{rapport.tex}, et \texttt{ch.c} qui contiennent le squelette (vide)
du code et du rapport que vous devez rendre.

\subsection{Remise}

Pour la remise, vous devez remettre deux fichiers (\texttt{ch.c} et
\texttt{rapport.tex}) par la page Moodle (aussi nommé StudiUM) du cours.
Assurez-vous que tout compile correctement sur \texttt{frontal.iro}.

\section{Détails}

\begin{itemize}
\item La note sera divisée comme suit: 20\% pour chacune des
  4 fonctionalités, et 20\% pour le rapport.
\item Tout usage de matériel (code ou texte) emprunté à quelqu'un d'autre
  (trouvé sur le web, ...) doit être dûment mentionné, sans quoi cela sera
  considéré comme du plagiat.
\item Le code ne doit en aucun cas dépasser 80 colonnes.
\item Vérifiez la page web du cours, pour d'éventuels errata, et d'autres
  indications supplémentaires.
\item La note est basée d'une part sur des tests automatiques, d'autre part
  sur la lecture du code, ainsi que sur le rapport.  Le critère le plus
  important, et que votre code doit se comporter de manière correcte.
  Ensuite, vient la qualité du code: plus c'est simple, mieux c'est.
  S'il y a beaucoup de commentaires, c'est généralement un symptôme que le
  code n'est pas clair; mais bien sûr, sans commentaires le code (même
  simple) et souvent incompréhensible.  L'efficacité de votre code est sans
  importance, sauf s'il utilise un algorithme vraiment particulièrement
  ridiculement inefficace.
\end{itemize}

\end{document}
